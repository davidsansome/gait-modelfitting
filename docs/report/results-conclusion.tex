\section{Conclusion}

From Table \ref{ClassificationResults2} we can see that there is one winning combination in terms of correct classification rates.
This combination is summarised in Table \ref{ConclusionTable}.

\begin{table}[hb]
	\centering
	\begin{tabular}{r|l}
		Search resolution & res$_\theta$ = 21, res$_\alpha$ = 11 \\
		Multiresolution search? & No \\
		Parameters to include in the DFT & Both thigh $\theta$ values \\
		Number of DFT components to use & All \\
		Normalise for mean? & Yes \\
		Normalise for variance? & No \\
		Distance measure & Euclidean distance in complex plane
	\end{tabular}
	\caption{Summary of the winning algorithm.}
	\label{ConclusionTable}
\end{table}

One interesting point regarding the results is that a higher search resolution, and correspondingly a closer fit to the manual fitting, did not result in a higher classification rate.
Intuition would suggest that as the resolution of our search increased the ``quality'' of the result would improve and the algorithm would be able to identify the subject with more accuracy.
In fact the opposite occured - the best performers at classification were consistently the ones with the lowest search resolution.

A tentative explanation for this is that as resolution of the search increases, the fitting process picks up more of the small high frequency variations and noise in the voxel data.
This in turn provides the DFT with extra irrelevant information, corrupting the results.
