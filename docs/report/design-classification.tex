\section{Classification}

After each frame of a sample has been through the modelfitting process, we can analyse the sequence as a whole
and produce a series of numbers which describe the subject's gait.
If we are successful, this series of numbers will be unique to each subject and can be used as the basis for classifying a new sample.


\subsection{Extracting a gait cycle}

Before we can perform any kind of analysis on a sample we need to identify and extract one gait cycle.
Figure TODO shows a typical plot of the change in the value of $\theta_1$ over time.
The graph resembles a sinosoid which we intuitavely know to be correct - when we walk our thighs swing forwards and backwards like a pendulum.
A gait cycle is usually defined as beginning and ending with heel strikes of the same foot (TODO: citation), which in Figure TODO would be two subsequent maximums of the sinosoid.
We however are going to begin our gait cycle when the thigh crosses the vertical, ie. when $\theta = 0$.

We can find this point where the thigh crosses the vertical by fitting a function $f_1(t)$ to our data points and finding where $f_1(t) = 0$.
The function we use is shown in Equation \ref{eqn:CurveFitting}:

\begin{equation}
	f_1(t) = d + a \sin(\frac{2 \pi t}{\xi} + \phi)
	\label{eqn:CurveFitting}
\end{equation}

The DC offset, $d$, and amplitude, $a$, were fairly similar for all subjects, so were kept fixed at $d = 0.05$ and $a = 0.55$.
Least squares regression was used to find the values for $\xi$, the period of the gait cycle, and $\phi$, the phase offset.

Equation \ref{eqn:CurveFitting} above describes the general motion of only one of the thighs.
We can increase accuracy by including the motion of the other thigh as well.
We make use of the observation that the motion of the two thighs are $90^\circ$ out of phase (see Figure TODO):

\begin{equation}
	f_2(t) = d + a \sin(\frac{2 \pi t}{\xi} + \phi + \pi)
	\label{eqn:CurveFitting2}
\end{equation}

The implementation used for our least squares fitting is similar to that in Listing \ref{leastsquarescode}, however instead of looking over a four-dimensional parameter space we are only interested in $\xi$ and $\phi$.
The ranges used for these variables are $\xi = [30, 40]$ and $\phi = [0, 2\pi]$.
The assumption that the period of one gait cycle lies in this range $[30, 40]$ was arrived at arbitrarilly, but holds true for all the subjects in the database.

The energy function used to determine the suitability of each set of parameters is worth mentioning as it has to take into account two curves and two sets of data.
It is shown in Listing \ref{curvefittingenergy}.
This function is used in a similar way to $modelEnergy$ in Listing \ref{leastsquarescode}.

\begin{lstlisting}[firstnumber=auto,language=c,morekeywords={step,function,foreach,in},frame=single,mathescape=true,caption={Energy function pseudo-code},label={curvefittingenergy},float=[tb]]
function energy($\xi$, $\phi$)
	$d = 0.05$
	$a = 0.55$
	
	$energy = 0$
	foreach $t$
		$expected_{left} = d + a \sin(\frac{2 \pi t}{\xi} + \phi)$
		$expected_{right} = d + a \sin(\frac{2 \pi t}{\xi} + \phi + \pi)$
		$actual_{left}$ = $\theta_1$[$t$]
		$actual_{right}$ = $\theta_2$[$t$]
		
		$energy$ += $(expected_{left} - actual_{left})^2$
		$energy$ += $(expected_{right} - actual_{right})^2$
	
	return $energy$
\end{lstlisting}

Once we have obtained values for $\xi$ and $\phi$ we can find the time of each zero-crossing like so:

\begin{equation}
	z_i = \frac{\xi \sin^{-1}(-\frac{d}{a} - \phi)}{2\pi} + i\pi
\end{equation}


\subsection{Discrete fourier transforms}



\begin{equation}
	X_i = \sum_{n=0}^{N-1} x_n e^{-\frac{2 \pi i}{N} i n} \quad \quad i = 0, \dots, N-1
\end{equation}


What does it do?

Implementation: FFTW library \cite{FFTW}
Assume that there's one period and the start is free of noise.  Ref set 4.

\subsection{Classification methods}
Show details of different classification approaches.
Graphs in complex plane to determine distance.

Take the polar distance instead!
