\section{Classification}

After every frame in a sample has been through the modelfitting process, we can analyse the sequence as a whole
and produce a series of numbers which describe the subject's gait.
If we are successful, this series of numbers will be unique to each subject and can be used as the basis for classifying any new samples.


\subsection{Extracting a gait cycle}

Before we can perform any kind of analysis on a sample we need to identify and extract one gait cycle.
This is important because we will be comparing different samples which might begin at different times during the subjects' gait cycle.

\begin{figure}[tb]
	\centering
	\includegraphics[width=\textwidth]{curvefitting.png}
	\caption{The change in $\theta_1$ over time for Sample 81.}
	\label{CurveFitting1}
\end{figure}

Figure \ref{CurveFitting1} shows a typical plot of the change in the value of $\theta_1$ over time.
The graph resembles a sinosoid which we intuitavely know to be correct - when we walk our thighs swing forwards and backwards like a pendulum.
A gait cycle is usually defined as beginning and ending with heel strikes of the same foot (TODO: citation), which in Figure \ref{CurveFitting1} would be two subsequent maximums of the sinosoid.
We however are going to begin our gait cycle when the thigh crosses the vertical, ie. when $\theta = 0$.

We can find this point where the thigh crosses the vertical by fitting a function $f_1(t)$ to our data points and finding where $f_1(t) = 0$.
The function we use is shown in Equation \ref{eqn:CurveFitting}:

\begin{equation}
	f_1(t) = d + a \sin\left(\frac{2 \pi t}{\xi} + \phi\right)
	\label{eqn:CurveFitting}
\end{equation}

The DC offset, $d$, and amplitude, $a$, were found to be fairly similar for all subjects, so were kept fixed at $d = 0.05$ and $a = 0.55$.
Least squares regression was used to find the values for $\xi$, the period of the gait cycle, and $\phi$, the phase offset.

\begin{figure}[tb]
	\centering
	\includegraphics[width=\textwidth]{curvefitting2.png}
	\caption{The change in $\theta_1$ and $\theta_2$ over time for Sample 81.
		The two parameters are $90^\circ$ out of phase.}
	\label{CurveFitting2}
\end{figure}

Equation \ref{eqn:CurveFitting} above describes the general motion of only one of the thighs.
We can increase accuracy by including the motion of the other thigh as well.
We make use of the observation that the motion of the two thighs are $90^\circ$ out of phase (see Figure \ref{CurveFitting2}):

\begin{equation}
	f_2(t) = d + a \sin\left(\frac{2 \pi t}{\xi} + \phi + \pi\right)
	\label{eqn:CurveFitting2}
\end{equation}

The implementation used for our least squares fitting is similar to that in Listing \ref{leastsquarescode}, however instead of looking over a four-dimensional parameter space we are only interested in $\xi$ and $\phi$.
The ranges used for these variables are $\xi = [30, 40]$ and $\phi = [0, 2\pi]$.
The assumption that the period of one gait cycle lies in this range $[30, 40]$ was arrived at by observation of the typical gait cycles in our database.

The energy function used to determine the suitability of each set of parameters is worth mentioning as it has to take into account two curves and two sets of data.
It is shown in Listing \ref{curvefittingenergy}.
This function is used in a similar way to $modelEnergy$ in Listing \ref{leastsquarescode}.

\begin{lstlisting}[firstnumber=1,language=c,morekeywords={step,function,foreach,in},frame=single,mathescape=true,caption={Energy function pseudo-code},label={curvefittingenergy},float=[tb]]
function energy($\xi$, $\phi$)
	$d = 0.05$
	$a = 0.55$
	
	$energy = 0$
	foreach $t$
		$expected_{left} = d + a \sin(\frac{2 \pi t}{\xi} + \phi)$
		$expected_{right} = d + a \sin(\frac{2 \pi t}{\xi} + \phi + \pi)$
		$actual_{left}$ = $\theta_1$[$t$]
		$actual_{right}$ = $\theta_2$[$t$]
		
		$energy$ += $(expected_{left} - actual_{left})^2$
		$energy$ += $(expected_{right} - actual_{right})^2$
	
	return $energy$
\end{lstlisting}

Once we have obtained values for $\xi$ and $\phi$ we can find the time $z$ of each zero-crossing like so:

\begin{equation}
	z_i = \frac{\xi \sin^{-1}\left(-\frac{d}{a} - \phi\right)}{2\pi} + i\pi
\end{equation}


\subsection{Discrete fourier transforms}

Now that we can extract a gait cycle from each sample we need to apply the DFT (Discrete Fourier Transform) to them to obtain information about the frequency components of the subject's gait.
Previous work (TODO: citations) have suggested that these frequency components can be used to form an identifying signature for the subject.

The DFT is defined by Equation \ref{eqn:DFT}.
A series $x$ of $N$ complex numbers is transformed into another series $X$ of $N$ complex numbers.
The number $X_i$ represents the $i^\text{th}$ multiple of the fundamental frequency (which is defined as $\frac{1}{N}$).

\begin{equation}
	X_i = \sum_{k=0}^{N-1} x_k e^{-\frac{2 \pi i}{N} i k} \quad \quad i = 0, \dots, N-1
	\label{eqn:DFT}
\end{equation}

Our input data to the DFT is purely real valued, so the output will obey the symmetry:

\begin{equation}
	X_i = X_{N-k}^*
\end{equation}

We can therefore disregard the second half of the output, as it will contain the same information as the first half.

The FFTW library \cite{FFTW} was used in the Modelfitter application to calculate the DFT.
Some typical results are shown in Figure \ref{FFTResults}.

\begin{figure}[tb]
	\centering
	\subfloat[Magnitude]{\includegraphics[width=5cm]{dft-leftthightheta-magnitude.png}}
	\quad
	\subfloat[Phase]{\includegraphics[width=5cm]{dft-leftthightheta-phase.png}}
	\caption{Results of taking the DFT of $\theta_1$ from one gait cycle of Sample 81.}
	\label{FFTResults}
\end{figure}


\subsection{Classification methods}

To correctly classify a new sample, we need some way of comparing it to the other samples already in the database.
The k-nearest neighbour algorithm with $k=1$ was chosen as it is simple to implement while providing good results.

A distance function was created to compare and evaluate the difference between two samples $s_i$ and $s_j$.
Each sample consists of $\frac{N}{2}$ complex numbers - the result of the DFT.

\begin{equation}
	d(s_i, s_k) \quad \quad i \neq k
\end{equation}

Several different implementations for this function were tested.
Details of each are given below, and quantative comparisons of their performance can be found in section TODO.

\begin{enumerate}
	\item Magnitude
\end{enumerate}




Magnitude-weighted phase
y/x
Real and imag
Magnitude excluding first
Magnitude-weighted phase excluding first
y/x excluding first
Real and imag excluding first

Normalized
Mean normalized
With other thigh
Higher res

A variety of methods for classifying the subjects were tried.
The aim of any method is to correctly identify 


Show details of different classification approaches.
Graphs in complex plane to determine distance.

Take the polar distance instead!
