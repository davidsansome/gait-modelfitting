\documentclass[a4paper,12pt]{report}
\usepackage[center]{caption}
\usepackage[pdftex]{graphicx}
\usepackage[pdftex]{hyperref}
\usepackage{amsmath}
\usepackage{subfig}
\usepackage{listings}

\lstset{breaklines=true,basicstyle=\small,tabsize=3,numbers=left,numberstyle=\tiny,numbersep=5pt,emptylines=1}

\lstdefinelanguage{Cg}
	{morekeywords={const,float,float2,float3,float4,float4x4,uniform,in,out,sampler3D,int,void,TEXCOORD0,COLOR0,TEXUNIT0,TEXUNIT1,cos,sin,return,for},
	sensitive=true,
	morecomment=[l]{//},
	}

\title{An Investigation into the Generation of 3D Skeletal Models from Voxel Data and their use in Gait Recognition}
\author{David Sansome}

\begin{document}
\bibliographystyle{plain}

\maketitle

\bigskip
\begin{abstract}
	This report describes the progress made into developing a 3D model-fitting application.
	The techniques of filter correlation and active contours are explored in detail, and their use in fitting a model to a reconstructed 3D image is examined.
	Stream-processor implementations are considered as a means of improving the performance of the algorithms.
	Preliminary results have been obtained from these methods, but there is much room for improvement.
	An extension to the filter correlation method is proposed that aims to resolve some of the problems observed during development.
\end{abstract}

\newpage

\tableofcontents

\chapter{Introduction and background}

\section{Project Goals}

\subsection{Model fitting}

The primary goal of the project is to produce an application that can process voxel data of a walking figure and generate a simple 3D skeletal model.
In its simplest form the model will consist of a pair of thighs and a pair of 

Create an application that processes voxel data of a walking figure, and generates a simple 3D skeletal model.
This model should feature at least the most basic joints (ankle, knee, hip),
with the angles between the limbs being derived from the data.

\subsection{Real-time}

Investigate ways to make the process above real-time,
or as close to real-time as is possible while analysing the motion of points over several frames.

\subsection{Gait recognition}

Produce a low-dimensional data-set from the model containing information that may be used to uniquely identify the person.

\subsection{More joints}

Investigate any benefit gained from including other joints in the skeletal model.

\section{Background and report of literature search}

This project builds heavily on previous research and development work by members of the University of Southampton.
The sample data used was obtained from test subjects walking through the Southampton Gait Tunnel.
Their movements were recorded on a number of cameras and the images processed by segmentation and 3D reconstruction algorithms.

This section covers any other work that has been useful in the developement of the project.

\subsection{2D model-fitting}

In \cite{cardboardpeople} Michael et al. describe an approach to model fitting entitled ``Cardboard People''.
Limbs in the model are represented by rigid planar patches.
These are connected together in a chain structure, with each rectangular plane consisting of four articulated points.
While the research presents an interesting approach to model-fitting, the technique is strictly 2D and therefore not directly applicable to this project.

In \cite{GaitBook} Nixon et al. summarise the general approach to model-based gait recognition.
The important angles between the limbs for consideration are shown to be those along the direction of movement.
A Pose Evaluation Function (PEF) is described that can be used to improve and measure the error of a particular fitting of a model to an image.
Both boundary and region matching errors are considered when calculating this function, reducing the chance that a model segment will be placed in the empty space between two limbs.
A technique building on this is proposed in \ref{ImprovedCC}.

\subsection{Non-modelfitting ways}

\subsection{How modelfitting can be applied to gait}

Cadence and stride length \cite{GaitBook}

\subsection{Active contours}\label{ContourBackground}

In \cite{CohenBalloons}, Cohen presents a model for active contours that considers the curve as a ``balloon'' that is inflated by a balloon force.
This force takes the form:

\begin{equation}
	F = k_1 \vec{n}(s) - k\frac{\nabla P}{\|\nabla P\|}
	\label{CohenEquation}
\end{equation}

Where $\vec{n}(s)$ is the normal vector to the curve and $P$ is the image force.
This force in Equation \ref{CohenEquation} makes our contour expand like a balloon to fill a space, and stop when it encounters an opposing force from features of the image.

Gunn and Nixon present \cite{GunnSnakes} an additional force that cause an active contour to take the shape of a circle.
The force $\mathbf{e}_i$ for the snaxel $i$ at position $\mathbf{V}_i$ is calculated from the neighbouring points by:

\begin{equation}
	\mathbf{e}_i = \tfrac{1}{2}(\mathbf{V}_{i-1} + \mathbf{V}_{i+1}) - \mathbf{V}_i + \theta_i\tfrac{1}{2}\mathbf{R}(\mathbf{V}_{i-1} - \mathbf{V}_{i+1})
\end{equation}

Where $\mathbf{R}$ is a $+90^\circ$ rotation matrix.
$\theta$ is the angle that should exist between a snaxel and its neighbours in order for them to form a circular shape.
This angle is defined by $\theta = tan(\pi/N)$ where $N$ is the number of snaxels.



\subsection{Fitting a model using active contours}

The water thing

From paper \cite{ImageSegModels}
Active Shape Models  Proposed by 78, 79
Set of points defined at various features in the image.

A shape Y can be defined by Y(model) + (matrix of eigenvectors)(vector of weights)
Limits on the value of these weights

So what i'd need to do to train model:
For every frame in the video:
  normalize the position and scale
  mark on points of key features
Calculate covariance matrix
PCA to get first eignvalues of covariance matrix

Criticism:
Seems very generic.

\subsection{Gait Recognition}

Cunado et al. present a detailed background to the various gait recognition approaches in \cite{GaitModels}.
The paper describes which of the limbs provide the most useful information for gait.
The smaller and easily obscured body parts such as the ankle and pelvis tend to provide useful identifying information, however identifying these from noisy images can be very difficult.

The focus of the paper is on the angle made by the thigh with the hip, the thigh and calf being modelled as simple pendulae.

\subsection{GPU Processing}

Talk about standard Cg and then CUDA

Any use by gait in the past?


\chapter{Design approaches}

\section{Modelfitter application}
Why it was needed.
\section{Pre-processing}
Assumptions: legs start half way up.  Legs are always the same height (are they?  do we scale with height of person?)
\section{GPU convolution}
\section{Snakes}
\section{3D mesh fitting}
Why others were bad idea.
Refer to literature that uses edge matching points.
Show the model in blender.
Describe matrix transformations.
Describe least squares fitting.
Implementation details - mapreduce, lookup caching, multiresolution.  Show table of times
\section{Classification}
Talk about FFTs.
How do we decide where to start from?  Curve fitting.
Assume that there's one period and the start is free of noise.  Ref set 4.
Show details of different classification approaches.
Graphs in complex plane to determine distance.

\chapter{Results}

\section{Accuracy of modelfitting}
Discuss resolution challenges.
Alpha error.
Comparison with manually fit data.
Show some graphs of theta over time.

\section{Accuracy of classification}
Show results of all approaches.
How does adding more limbs help?

\section{Comparison}
How does this compare with previous approaches?

\chapter{Future work}
Feet.

\section{Future Work}


\newpage
\bibliography{../../bibtex}

\newpage
\appendix

\chapter{User manual}

\section{Using the application}
\section{Command line interface}
Qt Patch ref dev docs

\chapter{Developer documentation}

\section{Pre-requisites and compilation}
SVN
Qt Patch
\section{Code structure}
UML

\chapter{Code Listings}

\section{ThighFilter.m}
\lstset{language=Matlab}
\lstinputlisting[label=ThighFilter.m]{"../oldsource/ThighFilter.m"}

\newpage
\section{correlation.cg}
\lstset{language=Cg}
\lstinputlisting[label=correlation.cg]{"../oldsource/convolution.cg"}

\end{document}
