\documentclass[a4paper,12pt]{article}
\usepackage[center]{caption}
\usepackage[pdftex]{graphicx}
\usepackage[pdftex]{hyperref}
\usepackage{amsmath}
\usepackage{subfig}
\usepackage{listings}

\lstset{breaklines=true,basicstyle=\small,tabsize=3,numbers=left,numberstyle=\tiny,numbersep=5pt,emptylines=1}

\lstdefinelanguage{Cg}
	{morekeywords={const,float,float2,float3,float4,float4x4,uniform,in,out,sampler3D,int,void,TEXCOORD0,COLOR0,TEXUNIT0,TEXUNIT1,cos,sin,return,for},
	sensitive=true,
	morecomment=[l]{//},
	}

\title{An Investigation into the Generation of 3D Skeletal Models from Voxel Data and their use in Gait Recognition}
\author{David Sansome}

\begin{document}
\bibliographystyle{plain}

\maketitle

\bigskip
\begin{abstract}
	This report describes the progress made into developing a 3D model-fitting application.
	The techniques of filter correlation and active contours are explored in detail, and their use in fitting a model to a reconstructed 3D image is examined.
	Stream-processor implementations are considered as a means of improving the performance of the algorithms.
	Preliminary results have been obtained from these methods, but there is much room for improvement.
	An extension to the filter correlation method is proposed that aims to resolve some of the problems observed during development.
\end{abstract}

\newpage

\tableofcontents
\section{Project Goals}

\subsection{Model fitting}

Create an application that processes voxel data of a walking figure, and generates a simple 3D skeletal model.
This model should feature at least the most basic joints (ankle, knee, hip),
with the angles between the limbs being derived from the data.

\subsection{Real-time}

Investigate ways to make the process above real-time,
or as close to real-time as is possible while analysing the motion of points over several frames.

\subsection{Gait recognition}

Produce a low-dimensional data-set from the model containing information that may be used to uniquely identify the person.

\subsection{More joints}

Investigate any benefit gained from including other joints in the skeletal model.

\section{Background and report of literature search}
\include{theory}
\include{implementation}
\include{results}
Feet.
Arms.
Rotation of torso.

\label{FutureWork:BaggyTrousers}
Leading edge would reduce problems related to clothing, as the leading edge is the one where the leg is.



\newpage
\bibliography{../../bibtex}

\newpage
\appendix

\section{Code listings}

\subsection{ThighFilter.m}
\lstset{language=Matlab}
\lstinputlisting[label=ThighFilter.m]{"../oldsource/ThighFilter.m"}

\newpage
\subsection{correlation.cg}
\lstset{language=Cg}
\lstinputlisting[label=correlation.cg]{"../oldsource/convolution.cg"}


\end{document}
