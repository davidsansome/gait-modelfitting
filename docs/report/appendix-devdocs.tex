\section{Pre-requisites and compilation}

The complete source code for the application is included on the DVD.
It is also available from the subversion repository at:

\texttt{http://svn.davidsansome.com/3yp}

\bigskip
\noindent Before compiling you should make sure you have the following libraries and tools installed:

\begin{enumerate}
	\item \textbf{CMake} (\texttt{http://www.cmake.org}).
	\item \textbf{Qt} (\texttt{http://www.trolltech.com}).
		Version 4.4 is \emph{required} as the modelfitter application uses the new QtConcurrent framework.
		At the time of writing Qt 4.4 was in release-candidate stage.
	\item \textbf{SVL} (\texttt{http://www.cs.cmu.edu/\~ajw/doc/svl.html}).
		I have created an Ubuntu package for SVL which is available from: \\
		\texttt{http://www.davidsansome.com/svl}.
	\item \textbf{ZLib} (\texttt{http://www.zlib.net/}).
	\item \textbf{FFTW} (\texttt{http://www.fftw.org/}).
\end{enumerate}

The application should compile cleanly on Linux, Mac OS X and Windows.
Windows users are advised to use MinGW instead of Visual Studio.
To build the application you should issue the following commands:

\begin{lstlisting}[firstnumber=1,language=sh,frame=single,morekeywords={cmake,make}]
cd modelfitting/bin
cmake ..
make
./modelfitting
\end{lstlisting}

\section{Code structure}
UML