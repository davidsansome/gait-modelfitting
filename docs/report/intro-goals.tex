\section{Project goals}

\subsection{Model fitting}
\label{Goals:ModelFitting}

The primary goal of the project is to produce an application that can process voxel data of a walking figure and generate a simple 3D model.
In its simplest form the model will consist of four cylinders representing the thighs and lower legs.
The angles these cylinders make with each other and the hips will be measured and stored to be used as part of the figure's identifying information.

The application should also be able to track the movement of the parts of the model over subsequent frames of a video.
It might take advantage of the temporal continuity of human gait - that the change in position of a certain point on a person's torso between frames is quite small.
This can be used to simplify model fitting across multiple frames -
the position of the model in the last frame can be used as a starting point and can narrow down the possible model positions of subsequent frames.


\subsection{Speed of execution}

One of the most interesting aspects of gait-recognition is its applicability to security, and the advantages of being able to identify people from a distance without their knowledge.
For gait systems to be of use in real-world environments they should make use of efficient image-analysis techniques and be able to identify people quickly.
A system should be able to raise an alert as soon as possible after a target figure moves into the surveillance area, and not be dependent on slow image-processing implementations.

While it might not be possible to make this process truly real-time, it is one of the goals of this project to investigate ways to improve the performance of gait and its ability to produce results more quickly.


\subsection{Gait recognition}

The eventual goal of this work on model fitting is to be able to identify and recognise the gait of figures captured in video.
This can be done by producing a signature for the subject and comparing this against other signatures of known individuals stored in a database.

