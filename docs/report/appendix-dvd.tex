The DVD contains an exported copy of the Subversion repository as well as precompiled binaries of the Modelfitting application for the three supported platforms.

Important directories and descriptions of their contents are listed below in Figure \ref{dvdlisting}.

\begin{table}[thb]
	\centering
	\begin{tabular}{|l|p{8cm}|}
		\hline
		Directory & Description \\
		\hline
		/binaries & Binaries for the main modelfitting application.
			The packages provided are: an Ubuntu Hardy Heron .deb, a Mac OS X disk image and a Windows .zip. \\
		/contourtest & Application to test active contours.
			Use qmake to compile, and press spacebar to advance the contour. \\
		/docs & The \LaTeX source files for the brief, interim and final reports.
			Also contains some of the referenced papers. \\
		/gaitdata & The voxel data used throughout the project.
			The different subdirectories contain fitted model data from different algorithms. \\
		/imgprocessingtest & Inital investigation into image processing on the GPU.
			Use CMake to compile.
			Requires NVIDIA Cg toolkit. \\
		/lookupgenerator & Utility to generate the lookup table for mesh searching.
			This table is found in /modelfitting/data/meshsearch.lut.
			Use qmake to compile. \\
		/manualfitcomp & Utility that compares the results of a fitting algorithm to manually labelled data.
			Uses the .dat files produced by the \emph{Plot params over time} tool and outputs a \LaTeX table.
			Use CMake to compile. \\
		/modelfitting & The source code for the main modelfitting application.
			This directory also contains the source for the Vis4D library written by Richard Seely.
			Compilation instructions are provided in Section \ref{devdocs} \\
		\hline
	\end{tabular}
	\caption{DVD listing.}
	\label{dvdlisting}
\end{table}
